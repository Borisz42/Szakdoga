\chapter{Fejlesztői dokumentáció} 
\label{ch:impl}

Az alkalmazás \textbf{Microsoft Visual Studio} segítségével készült. A kódok elsősorban C++, másodsorban -- a shaderek -- GLSL nyelven íródtak. Ha újra akarnánk fordítani akkor a \textbf{C:/} helyre csomagoljuk ki a mellékelt OGLPack.zip állományt (ez az szükséges csomagokat és függőségeket tartalmazza), majd futtassuk a \textbf{subst T: C:/} parancsot. Ezután megnyithatjuk a \textbf{.vcxproj} projektfájlt.

\section{Fejlesztői eszközök}

A program írása során számos könyvtár és API felhasználásra került. Ebben az alfejezetben ezek kerülnek bemutatásra.

\subsubsection{Simple DirectMedia Layer (SDL)}
A Simple DirectMedia Layer (SDL) könyvtár egy crossplatform multimédiás könyvtár, ami alacsony szintű, hatékony hozzáférést ad audio, bemeneti (egér, billentyűzet, joystick), valamint grafikus (OpenGL-en keresztül GPU-hoz) eszközökhöz. Az alkalmazásban az SDL 2.0 van használva. \cite{SimpleDi88:online}

\begin{figure}[H]
	\centering
	\includegraphics[width=0.5\textwidth]{Sdl}
	\caption{Az SDL logója}
	\label{fig:Sdl}
\end{figure}

\subsubsection{OpenGL API}
Az OpenGL (Open Graphics Library) egy részletesen kidolgozott szabvány, melyet a Silicon Graphics nevű amerikai cég fejlesztett ki 1992-ben. Olyan API-t takar, amely segítségével egy egyszerű, szabványos felületen keresztül megvalósítható a grafikus kártya kezelése és a háromdimenziós grafika programozása. Az interfész több ezer különböző függvényhívásból áll, melynek segítségével a programozók szinte közvetlenül vezérelhetik a grafikus kártyát, segítségükkel 3 dimenziós alakzatokat rajzolhatnak ki, és a kirajzolás módját szabályozhatják. \cite{OpenGL–W93:online} 

\begin{figure}[H]
	\centering
	\includegraphics[width=0.5\textwidth]{Opengl}
	\caption{Az OpenGL logója}
	\label{fig:Opengl}
\end{figure}

\subsubsection{Dear ImGui}
A \textbf{``Parameters''} feliartú panel megjelenítése ennek segítségével lett megoldva. A Dear ImGui (ImGui) egy egyszerű grafikai interfész könyvtár C++ nyelvhez. Segítségével egyszerűen kezelhetünk gombokat, csúszkákat, egyéb beviteli eszközöket valamint könnyen megjeleníthetünk adatokat. Gyors és hordozható, nincs szükség külső könyvtára csak néhány szimpla forrás fájl beillesztésére. \cite{ocornuti13:online}

\subsubsection{GLM}

Az OpenGL Mathematics egy headerben importálható C++ könyvtár mely a GLSL specifikációira szabva (elnevezések, funkciók) tartalmaz számtalan matematikai funkciót. Segítségével a GLSL-ben használatos funkciók és típusok C++-ban is használhatóak. \cite{OpenGLMa34:online}

\begin{figure}[H]
	\centering
	\includegraphics[width=0.5\textwidth]{GLM}
	\caption{A GLM logója}
	\label{fig:GLM}
\end{figure}

\section{Képalkotási módszer}
\label{sec:kepmod}

A megjelenítés \textbf{Raycast} technika (\ref{fig:trace_diag}.~ábra) alkalmazásával történik, egy speciális implicit reprezentáción, a távolságfüggvényeken. Ehhez minden pixelre ki kell számolni egy sugár paramétereit. Ezen sugár és felület metszetét a \textbf{Sphere tracing} algoritmussal kapjuk meg. A felületi normálist numerikusan számítjuk ki, melynek segítségével a felület már könnyen árnyalható.

\begin{figure}[H]
	\centering
	\includegraphics[width=0.5\textwidth]{trace_diag}
	\caption{A raycast technika ábrázolása. \cite{FileRayt97:online}}
	\label{fig:trace_diag}
\end{figure}

A \textbf{sphere tracing} algoritmus egy speciális esete az úgynevezett \textbf{raymarching} algoritmusnak. Működéséhez a kirajzolni kívánt objektumokat távolságfüggvényekkel kell reprezentálni, mely a virtuális tér bármely pontjáról képes meghatározni hogy milyen messze van az objektum felületétől. Az algoritmus egy implementációja \aref{src:raymarch}. forráskód részletben látható.

\lstset{caption={A sphere tarcing algoritmust megvalósító kód}, label=src:raymarch}
\begin{lstlisting}[language={C++}]
float RayMarch(vec3 ro, vec3 rd) {
	float dist=0.0;    
    for(int i=0; i<MAX_STEPS; i++) {
    	vec3 p = ro + rd*dist;
        float dS = GetDist(p);
        dist += dS;
        if(dist>MAX_DIST || dS<SURF_DIST ) break;
    }    
    return dist;
}
\end{lstlisting}

Az algoritmus két paramétert igényel: egy pontot és egy vektort, melyek meghatározzák a sugár kezdőpontját ér irányát. Ezen irány mentén lépked folyamatosan mindig annyit amennyit amekkora a távolság a legközelebbi felülethez képest. Ezt addig csinálja míg már kellően közel lesz (dS<SURF\_DIST), vagy ha már túl messzire ment (dist>MAX\_DIST), vagy esetleg túl sokat lépett (i>MAX\_STEP). 


\begin{figure}[H]
	\centering
	\includegraphics[width=0.4\textwidth]{SphereTracing}
	\caption{Sphere tracing: A kamerából kiinduló fénysugár mindig csak annyit halad előre amekkora a hozzá legközelebb lévő felület távolsága \cite{Raymarch94:online}}
	\label{fig:SphereTracing2}
\end{figure}

Az alkalmazásban teljesítményjavítás érdekében másik feltétel (dS<0.01*dist/MAX\_DIST) lett használva a felületi távolsághoz, mely számításba veszi hogy mennyire messzire vagyunk a kamerától. Ha messzebb vagyunk akkor nincsen szükség akkora részletességre és a felületi távolság így lehet nagyobb is.

\subsection{Fénymodell}

Ha túl messzire ment a sugarunk (dist>MAX\_DIST), akkor egyszerűen a háttér színét adjuk a pixelnek. Egyéb esetben pedig kiszámoljuk az adott pontban a felületi normálist és a fénymodell segítségével megállapítjuk a pixel színét.

A használt fénymodell nagyon sokat számít a kirajzolt kép minőségén. Ezért sok idő és energia lett rászánva ennek megalkotására, a végeredményre Ingio Quilez munkássága \cite{iqShader82:online} nagy befolyást gyakorlot. Ezen fénymodell komponensei \aref{fig:lighting}.~ábrán láthatóak. Ezek különböző színenkénti súlyozással vett összege alkotja végső színáranyalatot, melyre még egy gamma korrekció és egy ködszerű effekt is került -- ez utóbbi arra szolgál hogy a viszonylag kicsi maximális távolságot leplezze.

\begin{figure}[H]
	\centering
	\subfigure[Objektumok alapszíne]{\includegraphics[width=0.3\linewidth]{col}}
	\hspace{1pt}
	\subfigure[Módosított ambiens fény]{\includegraphics[width=0.3\linewidth]{amb}}
	\hspace{1pt}
	\subfigure[Ambiens eltakarás]{\includegraphics[width=0.3\linewidth]{occ}}
	\vspace{1pt}
	\subfigure[Spekuláris]{\includegraphics[width=0.3\linewidth]{spe}}
	\hspace{1pt}
	\subfigure[Diffúz * Árnyék]{\includegraphics[width=0.3\linewidth]{shadif}}
	\hspace{1pt}
	\subfigure[Ellenfény]{\includegraphics[width=0.3\linewidth]{bac}}
	\vspace{1pt}
	\subfigure[Élfény]{\includegraphics[width=0.3\linewidth]{fre}}
	\hspace{1pt}
	\subfigure[Árnyékalapú tükröződés]{\includegraphics[width=0.3\linewidth]{shadom}}	
	\caption{A fénymodell különböző komponensei.}
	\label{fig:lighting}
\end{figure}


\section{Megvalósítás}

A program írása során a Számítógépes Grafika BSc gyakorlat tárgy honlapjának \cite{GrafikaB26:online} projektjei nyújtottak kiindulási alapot. Az SDL működtetése és számos alapvető függvényhívás lett belőlük felhasználva.

\subsection{CMyApp osztály}

Ez a fő osztály. A main.cpp ezen keresztül kezeli le az egér és billentyűzet bemeneteit, szimulálja a fizikát és továbbítja a shaderbe a paramétereket hogy megtörténjen a kirajzolás. A fontosabb függvények külön részletezésre kerülnek.

\begin{figure}[H]
	\centering
	\subfigure[A diagram felső fele]{\includegraphics[width=0.4\linewidth]{cmyapp1}}
	\hspace{1pt}
	\subfigure[A diagram alsó fele]{\includegraphics[width=0.4\linewidth]{cmyapp2}}
	\caption{A CMyApp osztálydiagramja (két részre szedve a hosszúsága miatt)}
	\label{fig:uml}
\end{figure}

\subsubsection{bool Init()}

Ebben inicializálódik minden aminek kell. Definiálunk két darab háromszöget melyek összerakva egy olyan négyzetlapot alkotnak ami az XY síkot (-1,-1)-től (1,1)-ig lefedik. Ezt a négyzetlapot fogjuk később a shaderben átszínezni hogy megkapjuk a képet. Az uniform változók memóriacímeit is itt határozzuk meg, valamint a mozgatható labdák pozícióját, méretét és sebességét tartalmazó tömbök is itt kapnak kezdőértékeket.

\subsubsection{void Update()}

Minden képernyőfrissítés előtt lefut, ebben történik a ütközések ellenőrzése és a labdák mozgatása a pozícióik frissítése által. 

A labdák befogott pozíciói -- amikhez akkor közelítenek ha lenyomjuk a szóköz billentyűt -- itt kerülnek meghatározásra a kamerát leíró adatok ismeretében. Meg tudunk határozni egy a kamerából előre, egy jobbra és egy felfelé irányba mutató egységhosszú vektort. Ezek megfelelő kombinációjával és egy időtől és a labdák darabszámától függő forgatási mátrix felhasználásával tudjuk a pozíciójukat egy a kamerához képest fix helyzetű kör mentén beállítani. Mivel az időtől is függ a forgatási mátrix, így ezen kör mentén folyamatosan keringnek.

A labdák hívása és kilövése is itt van megvalósítva. A \textbf{SPACE} nyomva tartásakor labdák az imént kiszámolt pozíciójának és a jelenlegi pozíciójának különbségének konstans-szorosát kapja meg sebességül. Ezáltal minél messzebb van a pozíciótól annál gyorsabban közeledik hozzá. Illetve csak a billentyű nyomva tartásakor frissül egy \textbf{shoot\_time} változóban az idő. 

Ha az aktuális idő és a \textbf{shoot\_time} közötti idő csak kis mértékben tér el akkor tudjuk hogy most lett felengedve a billentyű. Ekkor pedig a kamerából előrefelé mutató vektor konstans-szorosa adódik a labdák sebességéhez.

A labdák \textbf{ütközéseinek megállapításához} a kirajzoláshoz is használt távolságfüggvényt használjuk a vizsgált labda középpontjával. Ha ez a távolság kisebb mint a gömb sugara akkor tudjuk valamivel ütköztünk.

A helyes \textbf{viselkedés megállapításához} ismerünk kell az ütközés pontjában a felületi normálist. A normális kiszámolásához használt módszer (\ref{src:norm}. kódrészlet) a távolságfüggvényt használja így egy egyszerűsítést tehetünk: az ütközési pont helyett a gömb középpontjában számoljuk a felületi normálist! Ezáltal az éleken és sarkokon ahol felületi normális nem igazán értelmezhető, ott is jó viselkedést produkáló vektort fogjuk kapni.

\lstset{caption={A felületi normálist kiszámoló függvény}, label=src:norm}
\begin{lstlisting}[language={C++}]
glm::vec3 CMyApp::GetNormal(glm::vec3 p) {
	float d = GetDist(p);
	float e = 0.0005;
	glm::vec3 n = d - glm::vec3(
		GetDist(p - glm::vec3(e, 0, 0)),
		GetDist(p - glm::vec3(0, e, 0)),
		GetDist(p - glm::vec3(0, 0, e)));
	return glm::normalize(n);
}}
\end{lstlisting}

Ezután már csak a kapott normálisra kell tükröznünk a vizsgált labdánk sebességvektorát és a megfelelő komponenseit némileg csökkenteni, ezzel szimulálva hogy visszapattanáskor veszít egy keveset az energiájából -- ehhez is a normálist használhatjuk, azért kell komponensenként mert a valóságban ha például elrúgunk ívesen egy focilabdát, akkor annak az vízszintes irányú mozgási energiája jóval kevesebbet csökken visszapattanáskor mint a függőleges irányú. 

Fel kell arra is készülnünk ha a \textbf{labdánk belemegy egy másik objektumba}. Ez többnyire akkor történik meg ha nagyon gyorsan mozog a labda és két ellenőrzés között beleér, vagy ha a labda tartásakor direkt belevisszük egy objektumba. Ha benne vagyunk valamiben akkor frissítjük a pozíció értékét a normális irányába egy kis mértékben. Ezáltal ha esetleg belekerülne valamibe a labdánk akkor kijön belőle automatikusan.

\begin{figure}[H]
	\centering
	\includegraphics[width=0.8\textwidth]{dot}
	\caption{A skaláris szorzat előjele és a vektorok közötti szög összefüggése \cite{13DotPro51:online}}
	\label{fig:dot}
\end{figure}

Mivel tudjuk hogy akár egy másik objektumba is belemehet a labdánk ezért még egy esetre fel kell készülnünk: a másik objektumból kifelé jövet nem szeretnénk hogy ismét tükröződjön a sebességvektor, hiszen akkor állandóan oda-vissza tükröződne és sosem jutna ki a labda. Erre megoldás hogy csak akkor tükrözzük a sebességvektort ha a normálvektorral bezárt szöge \textbf{tompaszög}. Ehhez mindössze a két vektor skaláris szorzatát kell venni és ellenőrizni hogy az eredmény negatív-e.

 
\subsubsection{void Render(int WindowX, int WindowY)} 

Paraméterként megkapja az ablak méreteit. Az imgui panel ebben a függvényben hívódik meg és töltődik fel a megjelenített tartalommal, illetve itt kapják meg a felületen bevitt értékeket a megfelelő változók. Itt hívódik meg továbba a shader is, melynek rengeteg uniform válozót adunk át:
\begin{compactitem}
	\item \textbf{WindowX, WindowY} - az ablak méretei
	\item \textbf{eye, at, up} - a kamerát definiáló vektorok
	\item \textbf{rtime} - az indítás óta eltelt idő másodpercben
	\item \textbf{multiBallPos} - a mozgó labdák koordinátáit és méreteit tartalmazó \textbf{tömb}
	\item \textbf{shift\_x, shift\_y, shift\_z} - az eltolás transzfomácó paraméterei
	\item \textbf{fold\_x, fold\_y, fold\_z} - a tükrözés tranformációk paraméterei
	\item \textbf{rot\_x, rot\_y, rot\_z} - a forgatás traszformáció paraméterei
	\item \textbf{iterations} - a fraktál iterációinak száma
	\item \textbf{ballCount} - a mozgó labdák száma
	\item \textbf{zoom} - a kamera látószögét befolyásoló paraméter
\end{compactitem}

A kép megalkotása teljes egészében a shaderben történik. Miután megtörtént az imgui panel konfigurásála és az uniform változók átadása, a \textbf{glDrawArrays()} függvény hívásával a vertex shaderbe juttatjuk az inicializáció során létrehozott és a futás során sehol nem módosított négyzetet.

Kezdetben egy négyzetet adtunk meg és ezt ablak formájúra kell nyújtani. A vertex shaderben az ablak méreteinek ismeretében kompenzáljuk a torzítást, azáltal hogy a továbbadott vektor x komponensét megszorozzuk az ablak szélességének és magasságának hányadosával. 

\begin{figure}[H]
	\centering
	\includegraphics[width=0.6\textwidth]{NDC}
	\caption{NDC (Normalized Device Coordinates) - az ablak koordinátáit normalizáltuk, hogy a ball alsó sarok (-1,-1), a jobb felső sarok (+1,+1) legyen. \cite{PythonOp51:online} }
	\label{fig:NDC}
\end{figure}

Ennek köszönhetően a fragment shaderben a bemeneti vektorunk egy olyan leképezés eredménye, ami az ablak minden pixeléhez hozzárendel egy -1 és 1 közötti (X,Y) koordinátapárost ahogy az \aref{fig:NDC}. ábrán is látható. Ezután a fragment sahderben \aref{sec:kepmod}. fejezetben ismertetett elmélet alapján megalkotársa kerül a kép.

\subsection{gCamera osztály}

Ez az osztály egy általános megvalósítása egy virtuális kamerának és közel egy-az egyben lett felhasználva a Számítógépes Grafika BSc gyakorlat tárgy honlapjának \cite{GrafikaB26:online} projektjeiből. 

Csak a kamerát mozgató függvényekre és az általuk befolyásolt eye, at és up vektorokra van szükség. Ezen három vektor segítségével lehet majd a shaderben megállapítani a sugarak irányát és kiindulási pontját.

\begin{figure}[H]
	\centering
	\includegraphics[width=0.6\textwidth]{gcam}
	\caption{A gCamera osztálydiagrammja}
	\label{fig:gcam}
\end{figure}

Ebben az osztályban valósul meg a virtuális kamera mozgatása, hiszen a helyváltoztatáshoz mindössze a kamerát leíró vektorokat kell megfelelően transzfomálni. Az ehhez szükséges függvények kerülnek részletezésre.

\subsubsection{void Update(float \_deltaTime)}

Ez a függvény minden kirajzolás előtt meghívódik és frissíti a kamerát definiáló vektorokat. Ehhez ismernünk kell az előre (\textbf{m\_fw}) és az oldalra mutató (\textbf{m\_st}) vektorokat melyeket a megfelelő szorzóval egyszerűen hozzáadunk az \textbf{eye} és \textbf{at} vektorainkhoz (\textbf{m\_eye} és \textbf{m\_at}) ahogyan azt \aref{src:updt}. kódrészletben is láthatjuk.

\lstset{caption={Az \textbf{eye} és \textbf{at} vektorok frissítése}, label=src:updt}
\begin{lstlisting}[language={C++}]
	m_eye += (m_goFw*m_fw + m_goRight*m_st)*m_speed*_deltaTime;
	m_at  += (m_goFw*m_fw + m_goRight*m_st)*m_speed*_deltaTime;
\end{lstlisting}

A szorzókat (\textbf{m\_goFw} és \textbf{m\_goRight}) a billenytűk fogják szabályozni. Megfigyelhetjük ha ezen változók értéke +1 vagy -1 akkor előre-hátra és jobbra-ballra is tudunk menni. Ellenben ha mindkettő nulla, akkor nem változnak az \textbf{eye} és \textbf{at} vektoriank, vagyis ekkor nem fog mozogni a virtuális kamera. 

Látható továbbá hogy a mozgás mértéke az \textbf{m\_speed} változótól és a \textbf{\_deltaTime-tól} fognak függeni. A felületen az \textbf{m\_speed} értékét állítjuk át az osztály \textbf{SetSpeed(float \_val)} függvényen keresztül, ezzel szabályozván a mozgási sebességet.


\subsubsection{void KeyboardDown(SDL\_KeyboardEvent\& key) }

A \textbf{WASD} és a \textbf{SHIFT} billenytűk segítségével tudunk mozogni a térben ezért ezeket a billenytűket folyamatason figyelni kell hogy le vannak-e nyomva. Ez egy switch és a paraméterül kapott esemény segítségével történik meg.

A \textbf{W} és \textbf{S} billentyűk lenyomására az \textbf{m\_goFw} változó értéke rendre 1 és -1 értéket kap. Az \textbf{A} és \textbf{D} billentyűk hatására pedig hasonlóképpen az \textbf{m\_goRight} változónak adunk 1 és -1 értéket.

A \textbf{SHIFT} billenytű lenyomására, ha az \textbf{m\_slow} változó éréke hamis, akkor az \textbf{m\_speed} változó értékét növeljük a négyszeresére, illetve az az \textbf{m\_slow-t} igazra állítjuk. Azért szükséges ezen változó figyelése mert csak egyszer szeretnénk megnégyszerezni a sebességet. 

\subsubsection{void KeyboardUp(SDL\_KeyboardEvent\& key) }

A billentyűk felengedését külön kell kezelni, ez ugyanolyan elven történik mint a \textbf{KeyboardDown()} esetében: A \textbf{W} és \textbf{S} billentyűk felengedésére az \textbf{m\_goFw} változó értéke 0 lesz. Az \textbf{A} és \textbf{D} billentyűk felengedésére pedig az \textbf{m\_goRight} változót nullázzuk.

A \textbf{SHIFT} billenytű felengedésére, ha az \textbf{m\_slow} változó éréke igaz, akkor az \textbf{m\_speed} változó értékét változtatjuk a negyedére, valamint az \textbf{m\_slow} hamis lesz.

\subsubsection{void UpdateUV(float du, float dv)}

Ez a függvény \aref{src:updt}. kódrészletben is látható \textbf{m\_fw} és \textbf{m\_st}, vagyis az előre és oladra mutató vektoroknak fog megfelelő értéket adni és frissíti az \textbf{m\_at} vektorunkat. Az osztály \textbf{MouseMove(SDL\_MouseMotionEvent\& mouse)} függvénye fogja megadni neki a \textbf{du} és \textbf{dv} paramétereket az egér függőleges és vízszintes elmozdulása alapján. Ez is minden kirajzolás előtt meghívódik.

\begin{figure}[H]
	\centering
	\includegraphics[width=0.5\textwidth]{polar}
	\caption{Polár koordinták \cite{Spherica72:online}}
	\label{fig:polar}
\end{figure}

Az \textbf{m\_at} vektorunkat (mely azt határozza meg hogy merre néz a kamera) leírjuk polárkoordinátákkal és a megfelelő szögekhez hozzáadjuk -- az értelmezési tarttományok figyelembe vételével -- a paraméterül kapott \textbf{du} és \textbf{dv} értékeket. Ezután átszámoljuk az így kapott vektort Descartes-féle koordinátákra és ez lesz a \textbf{m\_at} vektorunk új értéke.

Az \textbf{m\_fw} vetort az \textbf{m\_at} és \textbf{m\_eye} különbsége fogja adni, az \textbf{m\_st} vetort pedig a \textbf{m\_fw} és \textbf{m\_up} vektoriális szorzata.


\section{Tesztelés}

A tesztelés során megvizsgáltam hogy a funkciók (\ref{sec:ui}.~fejezet) az elvártaknak megfelelően működne-e. Az ehhez használt számítógép konfiguráció:
\begin{compactitem}
	\item Intel® Core™ i7-8700 CPU
	\item 16 GB RAM
	\item NVIDIA GeForce GTX 1660 GPU
	\item Windows 10 operációs rendszer
\end{compactitem}

\cleardoublepage
\subsection{Működés helyessége}

A következő viselkedéseket ellenőriztem:
\begin{compactenum}
	\item A virtuális kamerát lehet forgatni az \textbf{egérrel};
	\item A virtuális kamera nagyítását lehet állítani a \textbf{CTRL + görgővel} és a csúszkával is;
	\item A virtuális kamerát lehet mozgatni a \textbf{WASD} billentyűkkel;
	\item A mozgás sebességét lehet gyorsítani a \textbf{SHIFT} billentyűvel és állítani a görgővel vagy a csúszkával;
	\item Ha egyszerre gyorsítunk \textbf{SHIFT}-tel és görgővel, a görgő sebessége felülírja a shift gyorsítását;
	\item A fraktál paramétereit át lehet állítani a \textbf{csúszkákkal} és a fraktál ezen értékeknek megfelelően változik;
	\item A \textbf{``Zero values''} gomb hatására megközelíti minden érték a nullát;
	\item A \textbf{``Zero values''} gomb extrém érték esetén: 10000-re állítva egy csúszkát "nullázás" után az értéke 0.481 lett, 36-os iteráció mellett;
	\item A \textbf{``Random values''} gomb valóban véletlenszerű értékeket állít be. Időnként nem látható fraktálokat eredményez, minél nagyobb az iterácó értéke annál gyakrabban;
	\item A \textbf{SPACE} billentyű nyomva tartásakor a labdák megjelennek a felhasználó előtt. Ha közeli akadályba ütköztek akkor nem tudnak;
	\item A \textbf{SPACE} billentyű felengedésekor a labdák kilövődnek ha be van pipálva a \textbf{[shoot]}, és leesnek a gravitációnak megfelelően ha nincsen;
	\item A \textbf{labdák} ütközése tárgyakkal valósághűnek tűnik;
	\item A \textbf{labdák} ütközése egymással közepesen valóságosnak tűnik, a kezdőpozíció alatti félgömbhéjba folyamatosan eltolják egymást.
	\item A \textbf{labdák} legurulása lejtős felületeken lassabb mint kellene, közepesen valósághű.
	\item A \textbf{labdák} legurulása lejtős felületeken nem egyenletes, bizonyos irányú lejtők kevésbé működnek jól.
\end{compactenum}

\cleardoublepage
\subsection{Teljesítmény}

Az alkalmazás erősen GPU igényes.  A futás teljesítményét az alábbi kód segítségével teszteltem:
\lstset{caption={A tesztelést végző kód}, label=src:test}
\begin{lstlisting}[language={C++}]
if (TESTING)
	{
		if (delta_time_counter < avg)
		{
			delta_time_arr[delta_time_counter] = delta_time;
			++delta_time_counter;
		}
		else
		{
			delta_time_counter = 0;
			double avg_delta_time = 0.0;
			for (int i = 0; i < avg; ++i) { avg_delta_time += delta_time_arr[i]; }
			avg_delta_time /= avg;
			printf("Avrage of delta time: %f ms   Iterations: %d   Number of spheres: %d \n", avg_delta_time*1000, iterations, ballCount);
			iterations += 2;
		}
	}
\end{lstlisting}

A kód az \textbf{Update()} függvény alján található. A \textbf{delta\_time} két képfrissítés között eltelt időt jelöli. Ezen kód segítségével \textbf{avg} darab képfrissítési idő átlagát vesszük, ezt kiírjuk az aktuális iterációk száma és mozgatható gömbök száma mellett a terminálablakra, majd ez előbbi értékét megnöveljük kettővel. A tesztek avg=100 értékkel futottak.

A tesztelés idejére ki lett kapcsolva a \textbf{vsync}, hogy ne befolyásolja a mérést. Ha be lenne kapcsolva akkor az 1/60 s = 16.66 ms-nál kisebb kirajzolási idők eredményét nem tudnánk lemérni. Továbbá az a funkció is ideiglenesen ki lett kapcsolva ami alacsony képfrissítési ráta mellett kisebbre veszi az iterációk és gömbök számát.

A kezdeti pozícióhoz képest a kamera nem volt megmozdítva, valamint a tesztben érintett két paraméteren felül más nem lett átállítva a kezdeti alapértelmezettekhez képest. A futás eredményéből készült grafikon \aref{fig:Test1}.~ábrán látható.

\begin{figure}[H]
	\centering
	\includegraphics[width=0.8\textwidth,frame]{Test1b}
	\caption{Teljesítményteszt az iterációk számának függvényében}
	\label{fig:Test1}
\end{figure}


Megfigyelhető hogy az iterációk számától lineárisan függ a képfrissítési idő. Bármely két egymást követő sor különbsége 4-6 ms, illetve 10 iterációs eset ideje nagyjából fele a 20-nak és kb. harmada a 30-nak. Az is észrevehető hogy rettentően lelassítja az alkalmazást az iteráció növelése, elég 36-ig felmenni hogy a tesztelői környezeten 10 FPS alá essen a képfrissítési ráta (vagyis 100 ms fölé megy a frissítési idő), ami már határozattan nem folyamatos megjelenítést jelent.

Ha \aref{src:test}.~kód 15. sorát átírjuk \textbf{ballCount += 5} -re akkor azt is meg tudjuk vizsgálni hogy az mozgatható gömbök száma hogyan hat a képfrissítési időre. Ezen futás eredményéből készült grafikont \aref{fig:Test2}.~ábra mutatja.

\begin{figure}[H]
	\centering
	\includegraphics[width=0.8\textwidth,frame]{Test2b}
	\caption{Teljesítményteszt a gömbök számának függvényében}
	\label{fig:Test2}
\end{figure}

Itt is nagyjából lineáris összefüggést tapasztalunk, két egymás utáni érték között 1-3 ms eltérés mutatkozik, illetve az is látható hogy jóval kevesebb hatása van a gömbök számának növelése a teljesítményre. A tesztkonfigurációnak 0 iteráció mellett 40 gömb még nem okoz problémát, ellenben 0 gömb mellett 40 iteráció az előző teszt tanulsága szerint már sokkal inkább.

A gömbök esetében is inkább a megjelenítés okozza a gondot, a \textbf{glDrawArrays()} sor ideiglenes kommentezésével a kirajzolást lényegében megszüntetjük, ám az ütközések modellezését nem befolyásoljuk. Ha ezután újra futtatjuk ez előbbi tesztet, akkor megtudhatjuk hogy a fizika kiszámolása mennyire lassította a kirajzolást. A futás eredményének egy részéből készült grafikon \aref{fig:Test3}.~ábrán látható. 

A grafikonra 350-nél kevesebb gömbhöz tartozó mérési értékek nem szerepelnek, az azokhoz tartozó számolási idő kevesebb mint 1 ms volt. Ez jól mutatja hogy a kirajzolási időhöz képest az ütközések kiszámolása jelentéktelen.

\begin{figure}[H]
	\centering
	\includegraphics[width=0.8\textwidth,frame]{Test3b}
	\caption{Teljesítményteszt a gömbök számának függvényében, kirajzolás nélkül}
	\label{fig:Test3}
\end{figure}

Az ábra alapján viszont itt már inkább exponenciális összefüggést állapíthatunk meg lineáris helyett. Az utolsó érték 490 gömbbel már 1000 ms volt, de ki lett hagyva az ábrázolás megkönnyítése érdekében. A teszt alatt azonban a magasabb értékek során a processzor összesített kihasználtsága a Windows Task Manager szerint 14\%-os volt, és egyik szál sem mutatott állandó teljes terhelést. 

A fizika futása nem lett több szálra bontva, így nem meglepő hogy csak \textbf{egyetlen processzormagot használ} igazán, azonban az furcsa hogy azt az egyet nem teljes mértékben. A jelenség pontos forrása ismeretlen, könnyen lehet hogy a hardver vagy az operációs rendszer korlátozza biztonsági okokból, ellenben a probléma csak extrém körülmények között jelentkezik, így nem lett sok idő fordítva az ok felkutatására. 

A többi paraméter nincsen mérhető befolyással a teljesítményre. Fontos megjegyezni hogy a kirajzolás módja miatt a kamera pozíciója is befolyásolja a sebességet. Ha egyenesen felfelé nézünk, a sugár lépései jelentősen megnőnek, hiszen mindig a legközelebb lévő felület távolságát lépjük előre, így néhány iteráció elegendő ahhoz hogy kiléphessünk a maximális távolsággal a \textbf{RayMarch()} ciklusából.

Ugyanezen okból kifolyólag a talajtól közel kiinduló, azzal párhuzamos sugarak kis lépésekben tudnak csak haladni, így több iteráció szükséges a \textbf{RayMarch()} ciklusából való kilépéshez, ami lassabb kirajzolási sebességet eredményez.