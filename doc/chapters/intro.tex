\chapter{Bevezetés} % Introduction
\label{ch:intro}

Ha valaki játékfejlesztésbe szeretne kezdeni, akkor nagyon sok kihívással kell szembenéznie. Szerencsére manapság már levehet egy terhet a válláról ha egy előre megírt játékmotort (game engine) használ, amiből akár ingyenesen elérhetőt is lehet találni.

Egy játékmotor feladata hogy leegyszerűsítse a kirajzolást és az objektumok valósághű viselkedését. Ilyen viselkedés például, ha két szilárd tárgy ütközésekor azt várnánk el hogy azok ne menjenek bele egymásba, hanem inkább ténylegesen ütközzenek és "pattanjanak le" a másikról. Ezen viselkedés kiszámolása meglehetősen költséges tud lenni, ráadásul különböző alakzatoknál különböző algoritmusokat kell használni.

Szakdolgozatomban azzal foglalkozom hogy hogyan lehet a kirajzolást és az előbb említett valósághű viselkedést fraktálokkal elvégezni. Mivel a fraktálok nehezen leírható felülettel rendelkeznek így a lehető legáltalánosabban kell megközelíteni a velük való ütközést. 

Tekintve, hogy a kirajzoláshoz Sphere tracing módszert alkalmazok, így minden kirajzolt objektumomhoz van távolságfüggvényem. Ezek segítségével meg tudom állapítani a virtuális terem bármely pontjáról hogy az milyen messze van a kirajzolt felületektől. Ezen tudással nagyon egyszerűen és hatékonyan meg lehet állapítani hogy egy gömb ütközött-e bármivel, hiszen csak annyit kell ellenőriznünk hogy a gömb középpontja gömbsugárnyi távolságra van-e valamilyen felülettől. Ezután a gömb sebességvektorát a felület normálvektora körül megforgatjuk 180 fokban és ellentétes előjelűvé tesszük, mintha csak egy fénysugárra hatna a teljes fényvisszaverődés a felület normálisának megfelelő beesési merőlegesben.

Ezekhez azonban pontos és lehetőleg előjeles távolságfüggvények kellenek, így nem érdemes foglalkozni az olyan fraktálokkal amikhez a távolságfüggvény csak felső becslést ad. Ezért olyan jellegű fraktált használok mint a Sierpiński háromszög, amik IFS (Iterated Function System) által jönnek létre, azaz egy egyszerűbb alakzaton - aminek jól ismerjük a pontos távolságfüggvényét - sokszor végrehajtunk egymás után transzformációkat.
